\documentclass[twocolumn]{aastex61}
\pdfoutput=1 %for arXiv submission
\usepackage{amsmath,amstext}
\usepackage[T1]{fontenc}
\usepackage{apjfonts} 
\usepackage[figure,figure*]{hypcap}

\renewcommand*{\sectionautorefname}{Section} %for \autoref
\renewcommand*{\subsectionautorefname}{Section} %for \autoref

\shorttitle{AASTeX 6.1 Template}
\shortauthors{Author A et al.}

\begin{document}

\title{CIERA REU Template for the Astrophysical Journal with AASTeX 6.1}

\author{Author A}
\affiliation{Affiliation 1}
\affiliation{Affiliation 2}
\author{Author B}
\affiliation{Affiliation 1}
\affiliation{Affiliation 3}
\author{Author C}
\affiliation{Affiliation 3}
\author{Author D}
\affiliation{Affiliation 4}

\begin{abstract}
\begin{itemize}\itemsep0em
\item This is the most important part of the paper. 
\item No one reads a paper without first reading the abstract. 
\item Often only the abstract is read...
\item It must be short (less than 250 words typically)
\item But it must also tell the reader: 
\item What was Done,
\begin{itemize}\itemsep0em
\item What Hypothesis was being investigated
\item What was in fact measured
\item What Hypothesis (or model) was supported by your measurements
\item Why this is important, and broader impact
\end{itemize}
\item This is the hardest part of a paper to write, and should be done last.
\item Read other abstracts from the journal you are writing for to get an idea of the structure, and check journal websites.  (For instance, Nature abstracts have a fairly structured and specific format.)
\end{itemize}
\end{abstract}

\keywords{keyword1 --- keyword2 --- keyword3}

\section{Introduction}
\begin{itemize}\itemsep0em
\item This section should tell the reader the history of what other papers in this field have discovered
\item for example,
\begin{itemize}\itemsep0em
\item The age and distance to your star cluster
\item The mass, age, and evolutionary state of your variable star
\item The nature of the extrasolar planet and star that you studied
\end{itemize}
\item Be sure and check NASA ADS (or a paper service for your discipline) for any papers similar to yours or 
\begin{itemize}\itemsep0em
\item on your object (do this before writing the paper!)
\item Also be sure to check the arXiv as well (astroph for astronomers)
\end{itemize}
\item One should try and cite the major papers, especially the discovery papers (when relevant), that have led to the current understanding of the problem.
\item The introduction should very clearly spell out what the problem is you are writing about, and why it is important.
\item Overly long introductions should be avoided, but complete understanding of past papers should be presented.
\item The last paragraph in the Introduction outlines how your paper is written, section by section.
\end{itemize}

Here's an example of how to cite a paper \citep{1931ApJ....74...43H}.  Figure~\ref{f:CMDreg} is taken from \citet{2017ApJ...842....1G}.

\begin{figure}[!t]
\plotone{f1.eps}
\caption{
Here's a caption.
\label{f:CMDreg}
}
\end{figure}


\section{Observations or Simulations}
\begin{itemize}\itemsep0em
\item This is the easiest section of the paper to write (since you have done this)
\item For an observing paper, 
\begin{itemize}\itemsep0em
\item Clearly state what was observed 
\item Clearly note the telescope and/or instrument used. Cite the right paper for the instrument.  Include any obligatory footnotes.
\item Summarize the specs of the telescope and instrument (size of CCD, size of FOV, etc.), note size of scope.
\item Note the weather conditions, UT date/times, moon phase, airmass, seeing, clouds, etc.
\item Include all details about the observations, (e.g., total integration times, calibrations, filters used, wavelength range, etc.) 
\end{itemize}
\item For a theory paper,
\begin{itemize}\itemsep0em
\item Clearly state what was modelled
\item If you used others' software, cite their papers, but also give a brief summary of what the software does
\item If you wrote your own software, give a very detailed description (including equations when necessary, and hopefully with a link to a repository where readers can download and use your code).
\item Summarize all of the models that you generated (e.g., number of particles, relevant physical size and time scales, etc.)
\end{itemize}
\item Use a table to summarize the list of targets observed or models generated, with the relevant details
\item Note any special techniques that you used to optimize your science 
\end{itemize}


\section{Data Reduction (or Special Analysis Software?}
\begin{itemize}\itemsep0em
\item This section is mostly relevant to an observational paper, but in some theory papers one might make synthetic observations to compare directly with real observations.  In that case the following can apply to those synthetic observations.
\item This can be an easy section to write-up as well, if you have carried out the observations \& calibrations.
\item Be clear in how calibrations were used etc. to tell the reader how you removed the signatures of the telescope \& instrument 
\item Describe any software that was used here (and give appropriate references).
\item Clearly present your reduced data in some (or all) of the following:
\begin{itemize}\itemsep0em
\item In a table of photometry
\item A final reduced image of your target
\item A final spectrum (or time series) 
\end{itemize}
\item Always include a discussion on the errors and error analysis of your observations
\item By the end of this section the "science" is ready to be extracted from your tables, and/or figures.
\end{itemize}

\section{Analysis / Results}
\begin{itemize}\itemsep0em
\item This section is where you actually "extract" that science from your (reduced) data.
\item For example, 
\begin{itemize}\itemsep0em
\item from your table of B \& V photometry you can plot a Color-Magnitude diagram (with errors)
\item Or from a timeseries you can create a power spectrum (with errors, and pick out significant periods)
\item Or fit your planet transit data, smooth your data? and give errors.
\end{itemize}
\item For an observational paper, you may use someone else's theoretical models to fit the data and to derive difficult to measure quantities.  For a theory paper, you may instead compare with observations to show how your theory can give you these quantities. 
\item There should be a careful discussion of errors (in particular systematic errors for observations) here as well.
\item This section is normally where most of the plots and pretty figures will be placed.  All of these should be referenced in the text and explained in detail.  In most papers, we save the interpretation of these results for the next section.
\end{itemize}

For an example equation, please see Equation~\ref{e:poisson}.
\begin{equation} \label{e:poisson}
\Psi(t,\tau) = 1 - e^{-\left(t/\tau\right)}\sum_{x=0}^{n-1} \frac{\left(t/\tau\right)^x}{x!} ,
\end{equation}

For an example table, please see Table~\ref{clustab}.
%note you could include this as an external .tex files instead :
%\input{mytable.tex}
\begin{deluxetable*}{lccccccc}[!ht]
\tablecaption{Cluster Parameters \label{clustab}}
\tablehead{\colhead{Cluster} & \colhead{age} & \colhead{$M_\text{cl}$} & \colhead{$(M-m)_V$} & \colhead{$E(B-V)$} & \colhead{[Fe/H]} & \colhead{$n_{\rm H}$} & \colhead{$(r/r_c)_\text{max}$} \\
\colhead{} & \colhead{[Gyr]} & \colhead{[$M_\odot$]} & \colhead{} & \colhead{} & \colhead{} & \colhead{[10$^{20}$cm$^{-2}$]} & \colhead{} }
\tablewidth{0pt}
\startdata
NGC 188    &               6.2   &  1500 & 11.44   &   0.09   &             0.0  &  6.6 & 7.1 \\
NGC 2158   &                 2   & 15000 & 14.51   &   0.55   &            -0.6  & 41.9 & 7.9 \\
NGC 2682   &                 4   &  2100 &   9.6   &   0.01   &             0.0  &  3.3 & 7.2 \\
NGC 6791   &                 8   &  4600 & 13.38   &    0.1   &             0.4  & 10.7 & 10.6 \\
NGC 6819   &               2.4   &  2600 &  12.3   &    0.1   &             0.0  & 19.3 & 12.5 \\
NGC 7142   &               3.6   &   500 & 12.86   &   0.29   &             0.1  & 32.6 & 2.2 \\
\enddata
\tablenotetext{}{Some notes if necessary}
\end{deluxetable*}

For an example figure, please see Figure~\ref{f:CMDreg}.


\section{Discussion}
\begin{itemize}\itemsep0em
\item In this section one should try to relate how the quantities measured above relate to the scientific questions you are trying to answer.
\item Here you should be relating your work to the "big picture" presented in your introduction
\item How do you interpret the results from your analysis?  How do these results compare to other similar studies? What is new here? 
\item Why do your observations / models support one physical picture and not another?
\item For example, 
\begin{itemize}\itemsep0em
\item From a Color-Magnitude diagram, determine which stars are members, and size, age, and distance to the cluster
\item From the significant periods found from a variable star, understand the interior structure of the star and compare your results to other published ones
\item From the best fits of the transit model, determine the size of the planet in the system, and orbit 
\end{itemize}
\end{itemize}

\subsection{Future Observations / Simulations}
\begin{itemize}\itemsep0em
\item This section is optional (and could be included as a paragraph in either the Discussions or Conclusions section), but it helps point out where you see the field moving 
\item Note how your results could have been improved
\item Outline how much more could be gleaned from a better approach, or bigger scope, etc.
\end{itemize}


\section{Conclusions / Summary}
\begin{itemize}\itemsep0em
\item This section should written just before the abstract
\item It should be longer and a bit more detailed than the abstract
\item It should restate what your goals were (briefly)
\item What observations you made (briefly)
\item What your main measurements were (with error bars)
\item What physical model (hypothesis) is supported (and/or not supported) by your measurements and analysis
\end{itemize}

\acknowledgments
Here you thank all who helped make your paper possible (e.g., the telescope TAC and operators, the compute staff, other researchers who provided input but are not in the author list, etc.).  Thank any funding sponsor (like NSF, NASA etc.), and include the proper grant number and any specific wording specified by the agency (important to check on this and get it right!)

%optional additional bits to add
%\facility{facility ID}
\facilities{facility ID, facility ID, facility ID} 
\software{someCoolCode}

\bibliographystyle{yahapj}
\bibliography{references}

%do you need an appendix?
\appendix
\section{appendix section}

\end{document}